%%% Внесите свои данные - Input your data
%%
%%
\newcommand{\Author}{Д.А.\,Туников} % И.О. Фамилия автора 
\newcommand{\AuthorFull}{Туников Дмитрий Александрович} % Фамилия Имя Отчество автора
\newcommand{\AuthorFullDat}{Туникову Дмитрию Александровичу} % Фамилия Имя Отчество автора в дательном падеже (Кому? Студенту...)
\newcommand{\AuthorPhone}{+7-9XX-XXX-XX-XX} % номер телефорна автора для оперативной связи  
\newcommand{\Supervisor}{И.О.\,Яковлев} % И. О. Фамилия научного руководителя
\newcommand{\SupervisorVin}{И.О.\,Фамилию} % И. О. Фамилия научного руководителя  в винительном падеже (Кого? что? Руководителя ...)
\newcommand{\SupervisorJob}{должность} %
\newcommand{\SupervisorJobVin}{должность} % в винительном падеже (Кого? что?  Програмиста ...)
\newcommand{\SupervisorDegree}{степень} %
\newcommand{\SupervisorTitle}{звание} % 
%%
%%
%Руководитель, утверждающий задание
\newcommand{\Head}{И.О.\,Фамилия} % И. О. Фамилия руководителя подразделения (руководителя ОП)
\newcommand{\HeadDegree}{Должность руководителя}% Только должность:   
%Руководитель %ОП 
%Заведующий % кафедрой
%Директор % Высшей школы
%Зам. директора
\newcommand{\HeadDep}{M} % заменить на краткую аббревиатуру подразделения или оставить пустым, если утверждает руководитель ОП

%%% Руководитель, принимающий заявление
\newcommand{\HeadAp}{И.О.\,Фамилия} % И. О. Фамилия руководителя подразделения (руководителя ОП)
\newcommand{\HeadApDegree}{Должность руководителя}% Только должность:   
%Руководитель ОП 
%Заведующий кафедрой
%Директор Высшей школы
\newcommand{\HeadApDep}{O} % заменить на краткую аббревиатуру подразделения или оставить пустым, если утверждает руководитель ОП
%%% Консультант по нормоконтролю
\newcommand{\ConsultantNorm}{И.О.\,Фамилия} % И. О. Фамилия консультанта по нормоконтролю. ТОЛЬКО из числа ППС!
\newcommand{\ConsultantNormDegree}{должность, степень} %   
\newcommand{\ConsultantExtra}{И.О.\,Фамилия} % И. О. Фамилия дополнительного консультанта 
\newcommand{\ConsultantExtraDegree}{должность, степень} % 
\newcommand{\ConsultantExtraVin}{И.О.\,Фамилию} % И. О. Фамилия дополнительного консультанта в винительном падеже (Кого? что? Руководителя ...)
\newcommand{\ConsultantExtraDegreeVin}{должность, степень} %  в винительном падеже (Кого? что? Руководителя ...)
\newcommand{\Reviewer}{И.О.\,Фамилия} % И. О. Фамилия резензента. Обязателен только для магистров.
\newcommand{\ReviewerDegree}{должность, степень} % 
%%
%%
\renewcommand{\thesisTitle}{Тема выпускной квалификационной работы}
%\newcommand{\thesisDegree}{бакалавра}% магистра или специалиста% 
\newcommand{\thesisDegree}{работа бакалавра}% дипломный проект, дипломная работа, магистерская диссертация %c 2020
\newcommand{\thesisTitleEn}{Title of the thesis} %2020
\newcommand{\thesisDeadline}{дд.мм.202X}
\newcommand{\thesisStartDate}{дд.мм.202X}
\newcommand{\thesisYear}{202X}
%%
%%
\newcommand{\group}{N} % заменить вместо N номер группы
\newcommand{\thesisSpecialtyCode}{ХХ.ХХ.ХХ}% код направления подготовки
\newcommand{\thesisSpecialtyTitle}{Наименование направления подготовки} % наименование направления/специальности
\newcommand{\thesisOPPostfix}{YY} % последние цифры кода образовательной программы (после <<_>>)
\newcommand{\thesisOPTitle}{Наименование направленности (профиля) образовательной программы}% наименование образовательной программы
%%
%%
\newcommand{\institute}{
Название института
%Институт компьютерных наук и технологий
%Гуманитарный институт
%Инженерно-строительный институт
%Институт биомедицинских систем и технологий
%Институт металлургии, машиностроения и транспорта
%Институт передовых производственных технологий
%Институт прикладной математики и механики
%Институт физики, нанотехнологий и телекоммуникаций
%Институт физической культуры, спорта и туризма
%Институт энергетики и транспортных систем
%Институт промышленного менеджмента, экономики и торговли
}%
%%
%%




%%% Задание ключевых слов и аннотации
%%
%%
%% Ключевых слов от 3 до 5 слов или словосочетаний в именительном падеже именительном падеже множественного числа (или в единственном числе, если нет другой формы) по правилам русского языка!!!
%%
%%
\newcommand{\keywordsRu}{Стилевое оформление сайта, управление контентом, php, MySQL, архитектура системы} % ВВЕДИТЕ ключевые слова по-русски
%%
%%
\newcommand{\keywordsEn}{Style registration, content management, php, MySQL, system architecture} % ВВЕДИТЕ ключевые слова по-английски
%%
%%
%% Реферат не более 600 знаков на русский или английский текст
\newcommand{\abstractRu}{В данной работе были рассмотрены два алгоритма поиска пересечений линий на изображении: алгоритм на основе преобразования Хафа для прямых, и алгоритм с обучением машины опорных векторов. В качестве прикладной проблемы решалась задача поиска железнодорожных стрелок на изображениях, сделанных с головы локомотива. В процессе работы был создан датасет из 1357 размеченных железнодорожных стрелок. Данный датасет использовался для проверки качества работы обоих алгоритмов и обучения машины опорных векторов. 
Алгоритм на основе преобразования Хафа заключался в том, что изображение разбивалось на некоторое количество горизонтальных блоков, и в каждом блоке применялся алгоритм Хафа для поиска прямых, после чего пересечения линий искались с учетом геометрических особенностей различных видов железнодорожных стрелок. 
В алгоритме с использованием машины опорных векторов ключевым моментом было обучение SVM-классификатора, входными векторами которого были гистограммы ориентированных градиентов окрестностей стрелок, размеченных в датасете. Было обучено несколько классификаторов для поиска различных видов железнодорожных стрелок.
По результатам работы можно сделать вывод, что подход с машиной опорных векторов показал себя лучше с точки зрения точности обнаружения пересечений.} % ВВЕДИТЕ текст аннотации по-русски
%%
%%
\newcommand{\abstractEn}{This paper introduce two algorithms for searching the intersections of lines in an image: algorithm based on the Hough transform for straight lines and algorithm with training the support vector machine. As an applied problem, searching for railway arrows on images taken from the head of a locomotive was used. Dataset of 1357 labeled railway crosses was created. This dataset was used to check the quality of both algorithms and to train the support vector machine.
The algorithm based on the Hough transform consisted in the fact that the image was divided into a number of horizontal blocks and in each block the Hough algorithm was used to search for lines. After that the intersections of lines were searched for taking into account the geometric features of various types of intersections.
In the algorithm using the support vector machine, the key point was the training of the SVM classifier. The input vectors of SVM were histograms of the oriented gradients of the neighborhoods marked in the dataset of the railway arrows. Several classifiers were trained to search for different types of railroad switches.
According to the results of the work, we can conclude that the approach with the support vector machine has proved to be better in terms of arrows detection accuracy.} % ВВЕДИТЕ текст аннотации по-английски




%%% Не меняем дальнейшую часть - Do not modify the rest part
%%
%%
%%
%%
\newcommand{\HeadTitle}{\HeadDegree~\HeadDep}
\newcommand{\HeadApTitle}{\HeadApDegree~\HeadApDep}
\newcommand{\thesisOPCode}{\thesisSpecialtyCode\_\thesisOPPostfix}% код образовательной программы
\newcommand{\thesisSpecialtyCodeAndTitle}{\thesisSpecialtyCode~\thesisSpecialtyTitle}% Код и наименование направления/специальности
\newcommand{\thesisOPCodeAndTitle}{\thesisOPCode~\thesisOPTitle} % код и наименование образовательной программы
%%
%%
\hypersetup{%часть болка hypesetup в style
		pdftitle={\thesisTitle},    % Заголовок pdf-файла
		pdfauthor={\AuthorFull},    % Автор
		pdfsubject={Выпускная квалификационная работа \thesisDegree. Шифр и наименование направления подготовки: \thesisSpecialtyCodeAndTitle. \abstractRu},      % Тема
		pdfcreator={LaTeX, SPbPU-student-thesis-template},     % Приложение-создатель
%		pdfproducer={},  % Производитель, Производитель PDF % будет выставлена автоматически
		pdfkeywords={\keywordsRu}
}
%%
%%
%% вспомогательные команды
\newcommand{\firef}[1]{рис.\ref{#1}} %figure reference
\newcommand{\taref}[1]{табл.\ref{#1}}	%table reference
%%
%%
%% Архивный вариант задания ключевых слов, аннотации и благодарностей 
% Too hard to export data from the environment to pdf-info
% https://tex.stackexchange.com/questions/184503/collecting-contents-of-environment-and-store-them-for-later-retrieval
%заменить NewEnviron на newenvironment для распознавания команды в TexStudio
%\NewEnviron{keywordsRu}{\noindent\MakeUppercase{\BODY}}
%\NewEnviron{keywordsEn}{\noindent\MakeUppercase{\BODY}}
%\newenvironment{abstractRu}{}{}
%\newenvironment{abstractEn}{}{}
%\newenvironment{acknowledgementsRu}{\par{\normalfont \acknowledgements.}}{}
%\newenvironment{acknowledgementsEn}{\par{\normalfont \acknowledgementsENG.}}{}


%%% Переопределение именований %%% Не меняем - Do not modify
%\newcommand{\Ministry}{Минобрнауки России} 
\newcommand{\Ministry}{Министерство науки и высшего образования Российской~Федерации} %с 2020
\newcommand{\SPbPU}{Санкт-Петербургский политехнический университет Петра~Великого}
%% Пробел между И. О. не допускается.
\renewcommand{\alsoname}{см. также}
\renewcommand{\seename}{см.}
\renewcommand{\headtoname}{вх.}
\renewcommand{\ccname}{исх.}
\renewcommand{\enclname}{вкл.}
\renewcommand{\pagename}{Pages}
\renewcommand{\partname}{Часть}
\renewcommand{\abstractname}{\textbf{Аннотация}}
\newcommand{\abstractnameENG}{\textbf{Annotation}}
\newcommand{\keywords}{\textbf{Ключевые слова}}
\newcommand{\keywordsENG}{\textbf{Keywords}}
\newcommand{\acknowledgements}{\textbf{Благодарности}}
\newcommand{\acknowledgementsENG}{\textbf{Acknowledgements}}
\renewcommand{\contentsname}{Content} % 
%\renewcommand{\contentsname}{Содержание} % (ГОСТ Р 7.0.11-2011, 4)
%\renewcommand{\contentsname}{Оглавление} % (ГОСТ Р 7.0.11-2011, 4)
\renewcommand{\figurename}{Рис.} % Стиль СПбПУ
%\renewcommand{\figurename}{Рисунок} % (ГОСТ Р 7.0.11-2011, 5.3.9)
\renewcommand{\tablename}{Таблица} % (ГОСТ Р 7.0.11-2011, 5.3.10)
%\renewcommand{\indexname}{Предметный указатель}
\renewcommand{\listfigurename}{Список рисунков}
\renewcommand{\listtablename}{Список таблиц}
%\renewcommand{\refname}{\fullbibtitle}
%\renewcommand{\bibname}{\fullbibtitle}

\newcommand{\chapterEnTitle}{Сhapter title} % <- input the English title here (only once!) 
\newcommand{\chapterRuTitle}{Название главы}          % <- введите 
\newcommand{\sectionEnTitle}{Section title} %<- input subparagraph title in english
\newcommand{\sectionRuTitle}{Название подраздела} % <- введите название подраздела по-русски
\newcommand{\subsectionEnTitle}{Subsection title} % - input subsection title in english
\newcommand{\subsectionRuTitle}{Название параграфа} % <- введите название параграфа по-русски
\newcommand{\subsubsectionEnTitle}{Subsubsection title} % <- input subparagraph title in english
\newcommand{\subsubsectionRuTitle}{Название подпараграфа} % <- введите название подпараграфа по-русски