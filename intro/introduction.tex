\chapter*{Введение} % * не проставляет номер
\addcontentsline{toc}{chapter}{Введение} % вносим в содержание

Каждый знаком с пазлами еще с детства: имея n неперекрывающихся частей картинки необходимо восстановить исходное изображение, опираясь на их форму и цвета. В настоящее время решение данной задачи приносит большую пользу в различных областях биологии[2], так же существуют применения в дешифровании речи[3], археологии[4] и восстановления документов и фотографий[6]. Впервые данная проблема упоминалась картографом Джоном Спилсбури в 1760-е годы. Тем не менее первая попытка вычислительно решить данную проблему была совершена лишь в 1964 году Г.Фриманом и Л.Гардером. [5] С того времени, взгляд на эту проблему сместился с подхода, основывающегося на форме частей к исследованию цветовой схожести. Поэтому на данный момент решение данной задачи включает  в себя функцию оценки схожести частей изображения и алгоритм сборки, основанный на значениях этой функции. А наилучшие результаты были достигнуты в улучшении функции оценки совместимости фрагментов пазла.