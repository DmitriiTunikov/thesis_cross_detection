%%% Титульник ВКР / Thesis title 
%%
%% добавить лист в pdf-навигацию 
%% add to pdf navigation menu
%%
\pdfbookmark[-1]{\thesisTitle}{tit}
%%
\thispagestyle{empty}%
\makeatletter
\newgeometry{top=2cm,bottom=2cm,left=3cm,right=1cm,headsep=0cm,footskip=0cm}
\savegeometry{NoFoot}%
\makeatother



{\centering%
	Министерство науки и высшего образования Российской Федерации\\
	Санкт-Петербургский политехнический университет Петра Великого\\
	{%\bfseries %2020 - указание на изменения, которые могут быть введены в 2020 году
	Высшая школа прикладной математики и вычислительной физики}
\par}%


\vspace{0pt plus1fill} %число перед fill = кратность относительно некоторого расстояния fill, кусками которого заполнены пустые места


\noindent
\begin{minipage}{\linewidth}
	\vspace{\mfloatsep} % интервал 
	\begin{tabularx}{\linewidth}{Xl}
	&Работа допущена к защите     \\
	&Директор высшей школы     \\			
	&\underline{\hspace*{0.1\textheight}} Уткин Л.В.    \\
	&<<\underline{\hspace*{0.05\textheight}}>> \underline{\hspace*{0.1\textheight}} 2020~г.  \\ 
	\end{tabularx}
	\vspace{\mfloatsep} % интервал 	
\end{minipage}


\vspace{0pt plus2fill} %


{\centering%
	
	\MakeUppercase{\bfseries{}Выпускная квалификационная работа} \\ 
	\MakeUppercase{дипломная работа}%


%\intervalS% %ОБЯЗАТЕЛЬНО ДОБАВИТЬ ОТСТУП, ЕСЛИ ХВАТАЕТ МЕСТА
{\centering%
	\MakeUppercase{\bfseries{Поиск пересечений линий на изображении}}%

}\par%

%\intervalS% %ОБЯЗАТЕЛЬНО ДОБАВИТЬ ОТСТУП, ЕСЛИ ХВАТАЕТ МЕСТА
%по специальности % для специалистов
\noindent	по направлению подготовки 01.03.02 Прикладная математика и информатика{}\\% для бакалавров и магистров 
%\noindent Направленность  % для специалистов
\noindent	Направленность (профиль) 01.03.02\_04 Системное программирование % для бакалавров и магистров
% Лучше по~профилю, но что делать, так составили Положение
\par%





\vspace{4mm plus2fill}%

\noindent
\begin{tabularx}{\linewidth}{lXl}
	Выполнил              &	   &             \\
	студент гр.~3630102/60401     &    & Туников Д.А.     \\[\mfloatsep]

	Руководитель 		  &    &             \\
	доцент,		  &    &             \\
	к.б.н. 	  &    & Шубников В.Г. \\[\mfloatsep]
	
\end{tabularx} %


%
\vspace{0pt plus4fill}% 


\begin{center}%
Санкт-Петербург\\
2020
\end{center}%
\restoregeometry
\newpage